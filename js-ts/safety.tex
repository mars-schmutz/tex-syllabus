\textbf{Safe lab procedures:} Safety guidelines help protect you from accidents and injury. They also help to
protect equipment from damage. Some of these guidelines are designed to protect the environment from
contamination caused by improperly discarded materials.
\par
\bigskip
\textbf{General safety:} Safe working conditions help prevent injury to people and damage to computer
equipment. A safe workplace is clean, organized, and properly lit. Make sure you understand and follow
safety procedures.
\par
\bigskip
Follow basic safety guidelines to prevent cuts, burns, electrical shock, and damage to eyesight. As a
best practice, ensure that a fire extinguisher and first-aid kit are available nearby. Please refer to the
following guidelines when working on a computer.
\par
\bigskip
\textbf{Electrical safety:} Follow electrical safety guidelines to prevent electrical fires, injuries, and fatalities
in the home and the workplace. Power supplies and CRT monitors contain high voltage. To avoid an
electrical shock and to prevent damage to the computer, turn off and unplug the computer before
beginning a repair.
\par
\bigskip
\textbf{Fire safety:} Fire safety guidelines protect lives, structures, and equipment. Fire can spread rapidly and
be very costly. Proper use of a fire extinguisher can prevent a small fire from getting out of control. When
working with computer components, be aware of the possibility of an accidental fire and know how to
react. Be alert for odors emitting from computers and electronic devices. When electronic components
overheat or short out, they emit a burning odor. If there is a fire, be familiar with the following safety
procedures.
\par
\bigskip
\begin{itemize}
    \item Locate and read the instructions on the fire extinguishers in your workplace before you have to use them.
    \item Always have a planned fire escape route before beginning any work.
    \item Never fight a fire that is out of control or not contained. Instead, exit the building quickly.
    \item Contact emergency services for help.
\end{itemize}
\par
\bigskip
Be familiar with the types of fire extinguishers used at Dixie Tech. Each type of fire extinguisher has
specific chemicals to fight different types of fires.
\par
\bigskip
\textbf{All safety concerns and incidents should be reported to your instructor(s). All incidents should
be reported to security ASAP (435-272-6684);} security is trained in first aid, CPR, and incident management.
\par
\bigskip
See the \textbf{"Job Related Health and Safety"} flier for more computer related information.
